
\section{Communication Client/Serveur}

Afin d'assurer la communication entre le client et le serveur, nous utilisons l'architecture \textbf{REST}. En effet, elle offre des frameworks d'utilisation qui permettent de gérer l'ensemble des requêtes de la meilleure manière qui soit.\\

En pratique, le serveur harbormaster inclus un serveur REST possédant deux méthodes que les clients peuvent appeler :
\begin{description}
	\item[getJeu] une méthode en \emph{GET} qui permet à un client de récupérer l'état complet du jeu (lors de la connexion ou en cas de désynchronisation par exemple).
	\item[action] une méthode qui permet d'effectuer une des actions prédéfinies du jeu : \emph{inscription}, \emph{desinscription}, \emph{verrouillageBateau}, \emph{modificationTrajectoire}.
\end{description}

La méthode \emph{action} prend en paramètre un objet de type \emph{ServerAction} qui permet d'encapsuler en un seul objet le type de l'action et les paramètres à transmettre.\\

Le serveur devait aussi pouvoir transmettre des messages au client. Nous avons donc installé un autre serveur REST sur celui-ci. Selon le même schéma, le serveur harbormaster peut alors envoyer des messages au client, par exemple pour l'informer du verrouillage d'un bateau.\\

Du point de vue de l'identification, nous avons fait le choix de nous baser sur l'adresse IP du client qui s'enregistre. Le nom d'utilisateur permet de gérer ses scores.

\section{Diagramme de package}

La figure \ref{fig:package} représente le diagramme de package de notre application. 
\begin{figure}[htbp]
  \centering
 \includegraphics[width=\textwidth]{./images/Diagramme_package} 
  \caption{Diagramme de package}
  \label{fig:package}
\end{figure}


Le diagramme de classes initialement prévu est visualisable dans notre DC. 