
\section{Rappel du cahier des charges}

Pour rappel, le jeu devait comporter certaines contraintes de fonctionnement. Nous devions notamment veiller à respecter les restrictions suivantes:
\begin{description}
\item [Architecture distribuée] comportant un serveur, des clients et un client web;
\item[Multi-utilisateur] plusieurs joueurs (humain ou IA) peuvent participer à la même partie en même temps;
\item[Interfaces clientes génériques] possibilité de jouer avec divers périphériques d'entrée (souris, digitable \dots{});
\item[Sauvegarde des parties] afin de pouvoir les revisionner par la suite.
\end{description}

\section{Rappel des spécifications}


Le serveur permet de gérer le fonctionnement du jeu. Il assure la gestion de la carte, la trajectoire des bateaux, les actions de l'utilisateur et les connexions demandées par les clients.\\ 

Le client, quant à lui, permet d'afficher le jeu. Il s'agit donc d'un client léger, ce qui facilite la modification des périphériques d'entrée, et en fait un client générique.\\

La figure \ref{fig:casUtilisation} correspond au diagramme de cas d'utilisation de notre application. 


\begin{figure}[htbp]
  \centering
 \includegraphics[width=\textwidth]{./images/casDUtilisation} 
  \caption{Diagramme de cas d'utilisations}
  \label{fig:casUtilisation}
\end{figure}
