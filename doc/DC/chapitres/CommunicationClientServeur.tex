\section{Analyse des besoins}
Le protocole ou l'architecture utilisée doit être compatible avec les smartphones sous \emph{Android} et \emph{iOS} et avec l'ensemble des systèmes d'exploitation communs (\emph{Linux}, \emph{Mac OS X}, \emph{Windows}).\\

\begin{description}
	\item[Client :]~
	\begin{itemize}
		\item un client doit pouvoir s'enregistrer auprès du serveur pour jouer
		\item un client doit pouvoir jouer (verrouiller bateau, modifier trajectoire)
		\item un client doit pouvoir demander l'état complet du jeu pour se remettre à jour en cas de problème de communication ou lors de la première connexion
		\item un client doit pouvoir se désinscrire du serveur de jeu
	\end{itemize}
	\item[Serveur :]~
	\begin{itemize}
		\item un serveur doit pouvoir envoyer un message à tous les clients connectés (verrouiller un bateau, déverrouiller un bateau, transmettre un état du jeu)
	\end{itemize}
\end{description}

\section{Technologies étudiées}
\subsection{Sockets}
\begin{description}
	\item[Avantages]~
	\begin{itemize}
		\item Implémentation du minimum nécessaire : gain de rapidité
		\item Très bas niveau donc portable
	\end{itemize}
	\item[Inconvénients]~
	\begin{itemize}
		\item Difficulté de la mise en place
	\end{itemize}
\end{description}

\subsection{RMI}
\begin{description}
	\item[Avantages]~
	\begin{itemize}
		\item \emph{Callbacks} inclus dans la technologie
		\item Les objets sont situés côté serveur : gain de performance côté client
	\end{itemize}
	\item[Inconvénients]~
	\begin{itemize}
		\item Pas d'implémentation sur Android
	\end{itemize}
\end{description}

\subsection{Corba}
\begin{description}
	\item[Avantages]~
	\begin{itemize}
		\item \emph{Callbacks} inclus dans la technologie
		\item Les objets sont situés côté serveur : gain de performance côté client
	\end{itemize}
	\item[Inconvénients]~
	\begin{itemize}
		\item Difficulté de la mise en place
		\item Pas d'implémentation sur Android
	\end{itemize}
\end{description}

\subsection{WebServices}
\begin{description}
	\item[Avantages]~
	\begin{itemize}
		\item Portabilité
		\item Simplicité de mise en place
	\end{itemize}
	\item[Inconvénients]~
	\begin{itemize}
		\item Callbacks absents sans un serveur (ex : JBOSS)
	\end{itemize}
\end{description}

\subsection{EJB}
\begin{description}
	\item[Avantages]~
	\begin{itemize}
		\item \emph{Callbacks} inclus dans la technologie
	\end{itemize}
	\item[Inconvénients]~
	\begin{itemize}
		\item Pas d'implémentation sur Android
	\end{itemize}
\end{description}

\subsection{REST}
\begin{description}
	\item[Avantages]~
	\begin{itemize}
		\item Architecture de haut niveau : simplicité d'implémentation
		\item Framework complet dans \emph{Android} et \emph{Java}
		\item Framework compatible pour iOS (\emph{ASIHTTPRequest})
	\end{itemize}
	\item[Inconvénients]~
	\begin{itemize}
		\item Aucune connaissance de la vitesse de transmission de l'information
	\end{itemize}
\end{description}

\section{Choix de la technologie}
Au vu de l'ensemble des études menées (basées principalement sur les cours de l'éminent \emph{Alexandre Pauchet} et sur des recherches internet), nous allons utiliser l'architecture \emph{\textbf{REST}}.

En effet, elle est simple d'utilisation et nous offre des frameworks d'utilisation qui nous permettront de gérer l'ensemble des requêtes de la meilleure manière qui soit.

Nous avons soulevé le problème de la rapidité et il a été convenu avec notre tuteur que cette architecture ne serait à priori pas un frein à une exécution fluide.

















